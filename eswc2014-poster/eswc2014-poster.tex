%%%%%%%%%%%%%%%%%%%%%%%%%%%%%%%%%%%%%%%%%%%%%%%%%%%%%%%%%%%%%%%%%%%%%%%%%%%%%%%%%%%%%%%%%%%
%%%  What Are the Important Properties of an Entity? Studying the Knowledge Graph View %%%%
%%%%%%%%%%%%%%%%%%%%%%%%%%%%%%%%%%%%%%%%%%%%%%%%%%%%%%%%%%%%%%%%%%%%%%%%%%%%%%%%%%%%%%%%%%%

\documentclass[runningheads,a4paper]{llncs}

\usepackage[utf8]{inputenc}
\usepackage{amssymb}
\setcounter{tocdepth}{3}
\usepackage{graphicx}
\usepackage{tabularx}
\usepackage{url}
\usepackage{listings}
\usepackage{subfigure}
\usepackage{algorithmic}
\usepackage{algorithm}


\newcommand{\keywords}[1]{\par\addvspace\baselineskip
\noindent\keywordname\enspace\ignorespaces#1}

% todo macro
\usepackage{color}
\newtheorem{deflda}{Axiom}
\newcommand{\todo}[1]{\noindent\textcolor{red}{{\bf \{TODO}: #1{\bf \}}}}
\newcommand{\ghis}[1]{\noindent\textcolor{blue}{{\bf \{Ghislain}: #1{\bf \}}}}

% Language Definitions for Turtle
\definecolor{olivegreen}{rgb}{0.2,0.8,0.5}
\definecolor{grey}{rgb}{0.5,0.5,0.5}
\lstdefinelanguage{ttl}{
sensitive=true,
morecomment=[l][\color{brown}]{@},
morecomment=[l][\color{red}]{\#},
morestring=[b][\color{blue}]\",
}

%%%%%%%%%%%%%%%%%%%%%%%%%%%%%%%
%%%  Beginning of document  %%%
%%%%%%%%%%%%%%%%%%%%%%%%%%%%%%%

\begin{document}

% first the title is needed
\title{What Are the Important Properties of an Entity? Studying the Knowledge Graph View}

\author{Ghislain A. Atemezing\inst{1}, Ahmad Assaf\inst{1}, Rapha\"{e}l Troncy\inst{1} and Elena Cabrio\inst{1}\inst{2} }

\institute{EURECOM, Sophia Antipolis, France, \\
  \email{<atemezin@eurecom.fr>}
  \and INRIA, France, \email{<elena.cabrio@inria.fr>}
}

% a short form should be given in case it is too long for the running head
\titlerunning{What Are the Important Properties of an Entity?}	
%\authorrunning{Atemezing, Assaf, Troncy and Cabrio}	

\maketitle

%%%%%%%%%%%%%%%%%%
%%%  Abstract  %%%
%%%%%%%%%%%%%%%%%%

\begin{abstract}
In knowledge bases and more precisely in the Web of Data, entities have a lot of properties. A quick view to the different versions of DBpedia can give an idea of the phenomenon. However, it is still difficult to decide which ones are important than others depending on how we want to use these entities, such as for a visualization of some basic facts about the given entity. In this paper, we perform reverse engineering on the Google Knowledge graph panel to find out what are the properties shown according to the type of the entity. We compare the results obtained with users surveyed on some Entities. The preliminary results are promising as they shape the path towards a recommendation tool for detecting the core properties important to Entities.
\keywords{Crowdsourcing, Google Knowledge panel, visualization, scrapping, knowledge elicitation, intrinsic properties}
\end{abstract}

%%%%%%%%%%%%%%%%%%%%%%%%%
%%%  1. Introduction  %%%
%%%%%%%%%%%%%%%%%%%%%%%%%

\section{Introduction}
\label{sec:introduction}
\todo{rewrite this}
- 1) Motivation: in knowledge bases, entities have a lot of properties. Deciding which ones are more important than others depending on how we want to use these entities. Two use cases:
   . a) visualization of some basic facts about entities, for a multimedia QA system (QakisMedia) or for a second screen application (LinkedTV)
   . b) data integration (ontology matching), those properties having a bigger weights when computing alignments
 - 2) Approach 1: Google knowledge graph panel reverse engineering ... algorithms + first results
 - 3) Approach 2: User survey ... setup + results analysis
 - 4) Vocab for representing those "important" properties and dataset publication

\todo{@summarize the work of Thomas, Michiel Hildebrand, etc}

%%%%%%%%%%%%%%%%%%%%%%%%%%%%%%%%%%%%%%%%%%%%%%%%%%%%%%%%%%%%%%%%%
%%%  2. Reverse Engineering the Google Knowledge Graph Panel  %%%
%%%%%%%%%%%%%%%%%%%%%%%%%%%%%%%%%%%%%%%%%%%%%%%%%%%%%%%%%%%%%%%%%

\section{Reverse Engineering the Google Knowledge Graph Panel}
\label{sec:knowledge-graph}

 \todo{@Ahmda to Report the disambiguation strategy here } \\
 \todo{@Ahmad to report categories that do not have infoboxes while they still exist in Freebase. Run this will full dbpedia}
 
\begin{algorithm}[H]
\caption{Google knowledge graph panel reverse engineering Algorithm} \label{algoscrapping}
\begin{algorithmic}[1]
    \STATE INITIALIZE $equivalentClasses(DBpedia,Freebase) $ AS $vectorClasses$
    \STATE Upload $vectorClasses$ for querying processing
    \STATE Set $n$ AS number-of-instances-to-query
    \FOR { each $conceptType \in vectorClasses$}
	\STATE SELECT $n$ instances
	\STATE $listInstances \leftarrow$ SELECT-SPARQL($conceptType$, $n$)
		\FOR {each $instance \in listInstances$}
			\STATE CALL http://www.google.com/search?q=$instance$
			\STATE SCRAP GOOGLE KNOWLEDGE PANEL
			\STATE $gkpProperties \leftarrow$ GetData(DOM, EXIST(GKP))
			
		\ENDFOR
	\STATE COMPUTE ocurrences for each $prop \in gkpProperties$
    \ENDFOR
    \RETURN $gkpProperties$
\end{algorithmic}
\end{algorithm}


%%%%%%%%%%%%%%%%%%%%%%%
%%%  3. Evaluation  %%%
%%%%%%%%%%%%%%%%%%%%%%%

\section{Evaluation}
\label{sec:evaluation}


\subsection{User Survey}
\label{sec:survey}
We set up a survey\footnote{\url{http://eSurv.org?u=entityviz}} on the February 25th, 2014 for three weeks gathering the preferences of users in term of the properties they would like to be shown. We pick up nine entities per classes, namely \textsf{TennisPlayer}, \textsf{Museum}, \textsf{Politician}, \textsf{Company}, \textsf{Country}, \textsf{City}, \textsf{Film}, \textsf{SoccerClub} and \textsf{Book}.
We received quite a good number of participation (152 in total), with almost 72\% of users from academia, and 20\% coming from the industry. Generally, 94\% have heard about Semantic Web, and 35\% of the surveyed were not familiar with visualization tools. The detailed results\footnote{\url{https://github.com/ahmadassaf/KBE/blob/master/results/agreement-gkp-users.csv}} presents for each question in the file, the properties ranked by percentage received. We decide the more important properties to be the ones receiving more than for 10\% of the surveyed users.  
 For example, users surveyed don't care much about showing the \textsf{INSEE code} of a city, while they will love to see mostly \textsf{population}, \textsf{points of interest} properties. 

\subsection{Comparison with the Knowledge Graph}
\label{sec:comparison}
We limit the properties with moore than $10\%$ of answers from users surveyed. And on the Google Knowledge Panel (GKP) scrapping  results, we just pick the first top N ocurrences\footnote{The results show that N can be 4, 5 or 6} properties coming just after \texttt{label}, \texttt{type} and \texttt{properties}. These latter are called \textit{by-default-properties} as they are always presented in more than $98\%$ of the entities in the GKP. Table ~\ref{tab:agreement} presents for the $9$ classes surveyed the agreement percentage with the GKP. The highest agreement with \textsf{Museum}(66.97\%) while the lowest one for \textsf{TennisPlayer} (20\%) concept.   

 \begin{table}[!htp]
\centering{
\begin{tabular}{lccccccccc}
\hline 
 \textbf{Classes}	& TennisPlayer 	& Museum & Politician & Company & Country & City & Film & SoccerClub & Book	 \\ \hline 
\textbf{Agreement}& 20\%  & 66.97\% & 50\% & 40\% & 60\% & 60\% & 60\% & 50\% & 60\% \\ \hline
 
\\
\end{tabular}
\caption{Agreement on properties for 9 concept types, between users surveyed and Google Knowledge Panel.}
\label{tab:agreement}
}
\end{table}
With this set of 9 Concepts, We are covering $301,189$ entities of DBpedia that have keys in Freebase. And for each of them, we are empirically giving the most important properties are likely to be found in the ones where there is agreement between one of the biggest knowledge base (Google) and users preferences.

%%%%%%%%%%%%%%%%%%%%%%%%%%%%%%%%%%%%%%%%%%%%%%%%%%%%%%%%%%%
%%%  4. Modeling the Preferred Properties with Fresnel  %%%
%%%%%%%%%%%%%%%%%%%%%%%%%%%%%%%%%%%%%%%%%%%%%%%%%%%%%%%%%%%

\section{Modeling the Preferred Properties with Fresnel}
\label{sec:fresnel}


%%%%%%%%%%%%%%%%%%%%%%%%%%%%%%%%%%%%%%%
%%%  5. Conclusion and Future Work  %%%
%%%%%%%%%%%%%%%%%%%%%%%%%%%%%%%%%%%%%%%

\section{Conclusion and Future Work}
\label{sec:conclusion}
We have shown that it is possible to reveal important properties pf entities in a large knowlege base by starting comparing the properties obtained in the Knowledge Google Panel and the users preferences. We have provided an algorithm that for a given entity both in DBpedia and Freebase, compute the N ocurrences of the properties shown in the Google ``infobox''. We are sure these prelimary results can be benefit and helpful to decide which properties of an entity is worth for visualizing.  

For future work, we would improve the vocabulary used to describe the results obtained and make them available on a SPARQL endpoint. We are also investigating the use of Mechanical Truck to perform the survey for the rest of the classes and provide a complete comprehense dataset with the results obtained with the classes of DBpedia. 

%%%%%%%%%%%%%%%%%%%%%%%%%
%%%  Acknowledgments  %%%
%%%%%%%%%%%%%%%%%%%%%%%%%

\section*{Acknowledgments} \label{sec:acknowledgments}
This work has been partially supported by the French National Research Agency (ANR) within the Datalift Project, under grant number ANR-10-CORD-009. Elena is supported by the Labex project.

%+Labex (Elena)

\bibliographystyle{abbrv}
\nocite{*}
\bibliography{bibeswc}
\end{document}
